\def\thudbabelopt{english,italian}

\documentclass[target=bach,aauheader=,style=]{thud}

\course{Internet of Things, Big Data, Machine Learning}

\title{Sviluppo di un sistema per il monitoraggio di sensori per atleti}
\author{Alessandro Gaspari}
\supervisor{Prof. Ivan Scagnetto}
\cosupervisor{Dott.ssa Giulia Bongiorno \\ Dott. Luca Miceli}
\tutor{Ivan Scagnetto}



\usepackage[a-1b]{pdfx}
\usepackage[pdfa]{hyperref}
\usepackage{graphicx}
\usepackage{geometry}
\usepackage{subcaption}


\begin{document}
\maketitle

\begin{dedication}
    Dedica da scrivere

\end{dedication}

\acknowledgements
Ringraziamenti da scrivere.


%% Indice
\tableofcontents

%% Lista delle tabelle (se presenti)
%\listoftables

\listoffigures

%% Corpo principale del documento
\mainmatter

%% Parte
%% La suddivisione in parti è opzionale; solitamente sono sufficienti i capitoli.
%\part{Parte}

%% Capitolo
\chapter{Introduzione}

%% Sezione
\section{Ambito}
Negli ultimi anni, la raccolta dati tramite sensori IMU\footnote{Per IMU si intendono sensori che integrano internamente tre dispositivi di acquisizione attivi sugli assi $x$, $y$, $z$.} indossati da atleti ha subito una trasformazione significativa grazie all’evoluzione tecnologica dei microcontrollori e dei protocolli di comunicazione wireless, rendendo la misurazione dei parametri biomeccanici\footnote{I parametri biomeccanici sono delle misure usate per analizzare ed ottimizzare il movimento e la postura dell'atleta.} sempre più precisa, efficiente e integrabile in applicazioni di sensoristica avanzata. Questo progresso ha reso possibile una comprensione più approfondita dei movimenti corporei, aprendo la strada a sistemi di monitoraggio sofisticati che possono supportare allenatori, fisioterapisti e atleti nel miglioramento delle performance e nella prevenzione degli infortuni.

Questo progetto si colloca nell'ambito del monitoraggio sensoristico applicato al mondo dello sport, con particolare attenzione sia al tracciamento in tempo reale che all’analisi in differita dei dati raccolti. L'obiettivo è studiare in dettaglio i movimenti degli arti inferiori dell’atleta, analizzando in modo approfondito:

\begin{itemize}
    \item Movimenti sugli assi $x$, $y$, $z$ del ginocchio, per valutare flessibilità articolare e prevenire infortuni
    \item Movimenti sugli assi $x$, $y$, $z$ della tibia, per analizzare l’efficienza della falcata\footnote{Per falcata si intende il movimento di una gamba tra un appoggio al terreno e il successivo.} e la distribuzione delle forze  
    \item Pressioni plantari laterali, frontali e posteriori, utili per identificare squilibri posturali e ottimizzare la tecnica dell’atleta
\end{itemize}

Gli sport presi come riferimento sono il pattinaggio e la corsa, attività caratterizzate da movimenti complessi e intensi, che permettono la raccolta di numerosi dati in tempi relativamente brevi. L’analisi dei dati raccolti consente di ottenere metriche precise e affidabili, confrontabili tra atleti o sessioni differenti, fornendo così una base solida per valutazioni tecniche e studi comparativi tra varie attività dello stesso atleta o tra atleti con diversi livelli di abilità.
\section{Obiettivi}
Gli obiettivi principali del progetto sono:

\begin{enumerate}
    \item L'implementazione di un'applicazione mobile user-friendly\footnote{Con il termine user-friendly si intende un'interfaccia facile da utilizzare e intuitiva.} che consenta la profilazione dell'atleta, la connessione dei sensori e l'avvio (con il conseguente arresto) del tracciamento delle attività, con particolare attenzione alla facilità d’uso anche per utenti non esperti.
    \item Lo sviluppo di un sistema di memorizzazione sicuro e remoto, che garantisca la registrazione completa dei dati raccolti dai sensori, evitando la perdita di informazioni o conflitti tra dispositivi.
    \item La creazione di una dashboard\footnote{Con l'espressione dashboard si indica un'interfaccia grafica che mostra i dati in modo chiaro.} web reattiva che consenta la visualizzazione dei dati in tempo reale durante l’attività, o in differita tramite funzionalità di replay, rendendo immediata la comprensione delle performance e la possibilità di effettuare confronti tra diverse sessioni.
\end{enumerate}

Questi obiettivi non solo mirano a garantire il corretto funzionamento tecnico del sistema, ma anche a fornire strumenti utili per l’analisi avanzata delle performance, con una particolare attenzione all’usabilità e alla fruibilità dei dati.
\section{Risultati}
I risultati ottenuti confermano la validità delle scelte progettuali. L'applicazione mobile sviluppata è intuitiva e minimalista, permettendo all’atleta di concentrarsi sull’attività senza distrazioni. Il sistema di memorizzazione remoto assicura la registrazione completa e sicura dei dati, eliminando errori come lo scambio di nomi tra sensori o la perdita di informazioni. La dashboard web offre una rappresentazione chiara e fluida dei dati, consentendo agli utenti di monitorare performance e progressi in tempo reale, oppure di rivedere le sessioni passate per analisi dettagliate.
\section{Sinossi}
Questa tesi ha lo scopo di illustrare in dettaglio ogni fase dello sviluppo del sistema di monitoraggio proposto. Nei capitoli successivi, verrà esposto il problema principale affrontato con un confronto tra le soluzioni attualmente disponibili sul mercato, per poi proseguire con i requisiti tecnici e funzionali necessari allo sviluppo e le scelte progettuali implementate. Successivamente saranno inoltre trattati gli aspetti legati all’integrazione dei sensori, all’architettura dell’applicazione mobile e della dashboard web, con porzioni di codice opportunamente commentate, per comprendere al meglio le scelte progettuali attuate, verranno inseriti screenshot per mostrare i diversi casi d’uso. Infine, si presenteranno i risultati ottenuti, le possibili limitazioni e le opportunità di miglioramento e ottimizzazione del sistema per futuri sviluppi.
\chapter{Il problema}
\section{Esposizione del problema}
Nel contesto dell'attività sportiva moderna, l'analisi dei movimenti degli atleti risulta ormai indispensabile per correggere e migliorare notevolmente le abilità dell'agonista o dell'amatore. La raccolta dati tramite dispositivi di sensoristica sempre più avanzati, evidenzia quanto sia effettivamente critico il monitoraggio per evitare infortuni e per migliorare le performance.

\begin{figure}[htbp]
\centering
\begin{tabular}{cc}
\includegraphics[width=0.48\textwidth]{assets/PercentualeAdozione.png} &
\includegraphics[width=0.48\textwidth]{assets/GraficoMaratone.png} \\
(a) Adozione di sensori fra atleti & (b) Tempi mediani di arrivo nelle maratone
\end{tabular}
\caption{Confronto fra adozione di sensori wearable e performance nelle maratone.}
\label{fig:tabular}
\end{figure}

Come possiamo osservare a sinistra della figura \ref{fig:tabular}, all'inizio del 2010, gli sportivi, agonisti e non, che utilizzavano nei loro allenamenti dei sensori indossabili erano una piccolissima porzione rispetto al totale degli atleti. Infatti si parla di solamente un 2\%, secondo una stima basata su un'analisi di mercato dei fitness tracker in Nord America \cite{fortune2025wearables} e un'altra analisi di mercato in tutto il mondo \cite{grandview2024}.
Cinque anni dopo, nel 2015, si è osservato un incremento significativo nell'adozione dei sensori wearable, secondo anche un utile sondaggio effettuato fra gli allenatori americani \cite{luczak2020coaches}. Nel 2020 avvenne l'incremento più sostanzioso, portando più della metà degli atleti a utilizzare sensori durante i loro allenamenti per migliorare le prestazioni ed evitare infortuni dovuti a scorretti movimenti o appoggi. Ad oggi si stima che circa l'85\% di tutti gli atleti mondiali utilizzino sensori wearable per il monitoraggio avanzato durante le loro attività.

Proseguendo con l'analisi del secondo grafico, generato utilizzando i dati del World Athletics Database \cite{camminady2023wa}\footnote{Il World Athletics Database è una collezione di dati presente su GitHub con numerosi risultati di varie discipline sportive riguardanti l'atletica} possiamo invece notare come negli anni, sempre partendo dal 2010 e arrivando ad oggi, i tempi mediani con cui i maratoneti portavano a termine la loro corsa sono diminuiti considerevolmente, partendo da circa 129 minuti, arrivando a 127 negli ultimi anni, delineando un trend decrescente, che ci indica dunque come gli atleti riescano a mantenere una velocità più alta durante tutta la maratona, permettendogli di ottenere delle performance ottimali \footnote{Il picco negativo registrato nel 2020, è dovuto all'annullamento della maggior parte delle gare a causa del Covid-19, permettendo solo a professionisti di altissimo livello, la partecipazione ad una scarsa quantità di eventi}. L'analisi della maratona risulta migliore per comprendere al meglio quanto sia effettivamente efficace l'utilizzo dei sensori wearable durante gli allenamenti rispetto all'analisi di una corsa più breve, poichè in un tratto di pista maggiore, l'atleta dovrà sforzarsi per più tempo a mantenere una postura corretta, un appoggio del piede costantemente equilibrato e un movimento delle articolazioni coordinato, gli permetterà dunque di accumulare vantaggio data la lunga durata della gara. Questo miglioramento di performance risulterebbe minimo se non nullo in una gara così breve come i cento metri ad esempio, la cui ridotta lunghezza non consentirebbe all'atleta di accumulare vantaggio correggendo costantemente i suoi errori. L'altro vantaggio molto importante da considerare è sicuramente quello della prevenzione infortuni. In molti sport si osserva che, per sovraccarichi muscolari o appoggi squilibrati, si vanno a generare delle lesioni talvolta anche molto gravi negli atleti, costretti poi a dover intraprendere un percorso di riabilitazione. Secondo uno studio \cite{barca_innovationhub_2025} tenuto dal centro medico sportivo del famoso club calcistico Barcellona però, l'integrazione di dati biometrici e fisici raccolti attraverso sensori wearable, riescono ad anticipare e predire segnali di rischio infortunio, evitando così che l'atleta possa farsi del male, compromettendo la sua salute.

\begin{figure}[h]
    \centering
    \includegraphics[width=0.8\textwidth]{assets/InfortuniBarcellona.png}
    \caption{Infortuni stagionali del Barcellona FC pre e post adozione di sensori}
    \label{Infortuni_Barcellona} 
\end{figure}

Questo viene anche confermato dal grafico in figura \ref{Infortuni_Barcellona}, costruito basandosi sul dataset di Transfermarkt \cite{transfermarkt2023injuries}. Analizzando il grafico, notiamo come nel 2021, nel Barcellona si fosse registrato il maggior numero di infortuni per stagione. A seguito dell'adozione di sensoristica moderna e allenamenti mirati alla prevenzione di infortuni, come citato dal Barça Innovation Hub \cite{barca2021wearables}, il numero di infortuni diminuì notevolmente, rendendo profittevole l'investimento sulla sensoristica. Un altro importante studio
tenuto da Preatoni Ezio \cite{preatoni2022wearables} evidenzia come le tecnologie sensoristiche indossabili (e.g. accelerometri o giroscopi) riescano a registrare dei dati che consentano la successiva identificazione di alterazioni di movimento o di carichi di lavoro, associati al rischio di infortunio. 

Un importante studio specifico sul pattinaggio corsa \footnote{Il pattinaggio corsa è una specialità del pattinaggio a rotelle regolamentata a livello internazionale, praticata su piste indoor/outdoor o su strada.} ha analizzato l'efficacia di calze intelligenti dotate di sensori
piezoelettrici integrati direttamente nelle fibre tessili (posizionati su metatarso mediale, laterale e
tallone) e accelerometri per il monitoraggio in tempo reale del pattinaggio di velocità. Questi sensori
piezoelettrici rilevano le variazioni di pressione plantare, permettendo di distinguere i pattern motori
tra atleti esperti e neofiti in base a come il complesso piede/pattino veniva appoggiato a terra. Un
parametro aggiuntivo ma importante della ricerca trattata in tale articolo è la misura, mediante i due
accelerometri posizionati a livello dei malleoli esterni, del Bongiorno Index, un indicatore di
efficienza che calcola la percentuale di accelerazione latero-laterale rispetto all'accelerazione globale
impressa durante ogni falcata. Tale indice, già proposto in un precedente lavoro \cite{articlebongiorno1}, veniva in passato
ricavato mediante gli accelerometri di uno smartphone posizionato a livello di S1 \footnote{Per S1 si intende la prima vertebra sacrale in regione lombosacrale.}, ma il livello di
precisione raggiunto era inferiore, dipendendo dalla qualità del device/smartphone utilizzato e
potendo risentire di spostamenti del dispositivo stesso durante la pattinata. I risultati dello studio del
2025 mostrano che un atleta d'élite raggiunge un indice significativamente più alto (circa 58\%)
rispetto a un principiante (circa 27\%), definendo una "firma digitale" per l'ottimizzazione tecnica
dell'allenamento \cite{articlebongiorno2}.
Per poter però eseguire le attività di monitoraggio e analisi dei dati registrati dai sensori, si necessita di un sistema efficace e reattivo, facilmente utilizzabile dall'atleta. I sistemi generalmente richiesti per questi tipi di monitoraggi, necessitano di:

\begin{itemize}
   \item Un'applicazione installata sul dispositivo mobile mantenuto dall'atleta durante l'attività per abilitare la connessione ai sensori e l'avvio delle registrazioni.
   \item Una dashboard web che proceda all'elaborazione dei numerosi dati ricevuti, per permettere al soggetto analizzante di controllare tutti i parametri di interesse, in aggiunta alla posizione GPS del dispositivo mantenuto dall'atleta.
\end{itemize}

\section{Stato dell'arte}
\label{statoarte}
Attualmente sul mercato si trovano diversi tipi di applicazioni mobili che permettono il monitoraggio tramite sensoristica avanzata. Una delle più importanti da citare è sicuramente \textit{Sensoria Run} e \textit{Sensoria Lab}. Tramite queste due applicazioni scaricabili dai vari app store, ci si può connettere ai dispositivi Sensoria per effettuare l'abbinamento ad essi e successivamente avviare il tracciamento. Sensoria attualmente offre una vasta gamma di sensori compatibili con l'app, quali:

\begin{itemize}
    \item Calzini con IMU e sensori di pressione laterale, posteriore e frontale (a sinistra nell'immagine \ref{SensoriSensoria}).
    \item Magliette e canottiere con supporto per cardio-frequenzimetri (in mezzo alla figura \ref{SensoriSensoria}).
    \item Cardio-frequenzimetri con precisioni elevate.
    \item Smart Band da polso per misurazioni di parametri biometrici.
    \item Sensori con IMU per gambe o braccia con annessi supporti per l'uso (a destra nell'immagine \ref{SensoriSensoria}).
\end{itemize}

\begin{figure}[h]
    \centering
    \includegraphics[width=1\textwidth]{assets/sensoriasensori.png}
    \caption{Vari dispositivi indossabili del marchio Sensoria.}
    \label{SensoriSensoria}
\end{figure}

Una volta connessi i dispositivi desiderati e avviata l'attività desiderata che si è scelta tra le tante, verranno visualizzati sullo schermo vari dati utili all'atleta, osservabili anche dopo aver terminato l'attività. Difatti nell'app Sensoria Run vi è una schermata di archivio (a sinistra della figura \ref{SchermateSensoria}) tramite la quale è possibile riguardare ciascuna attività osservando grafici statici sulle performance dell'atleta durante il monitoraggio, in aggiunta a delle statistiche calcolate in media su tutto il percorso. Dalla dashboard web di Sensoria invece si può osservare il percorso effettuato su una mini mappa, con informazioni aggiuntive sul percorso, come dati sull'appoggio del piede, oppure informazioni su miglioramenti o peggioramenti nel tempo impiegato per compiere un determinato percorso già registrato in precedenza.

\begin{figure}[htbp]
\centering
\begin{tabular}{cc}
\includegraphics[width=0.18\textwidth]{assets/sensoriaRun.png} &
\includegraphics[width=0.48\textwidth]{assets/WebSensoria.png} \\
(a) Schermata di riepilogo attività in app & (b) Schermata web di riepilogo attività 
\end{tabular}
\caption{Schermate di dashboard web e app per il riepilogo attività.}
\label{SchermateSensoria}
\end{figure}

Tutte queste informazioni consultabili rendono il monitoraggio utile, dando la possibilità di correggere eventuali carichi troppo alti sulle articolazioni o appoggi squilibrati semplicemente osservando i grafici e le statistiche. Un'ulteriore dato molto importante, anche per la sicurezza dell'atleta, è il monitoraggio del battito cardiaco, infatti, osservando le variazioni di esso è possibile accorgersi di eventuali anomalie (come battito troppo elevato o troppo basso) che potrebbero indicare problemi cardiaci nell'atleta.  
Tramite l'altra applicazione ideata dagli sviluppatori di Sensoria, ossia Sensoria Lab, pensata principalmente per chi si occuperà dell'analisi approfondita dei dati (ricercatori, sviluppatori...), si potranno analizzare tutti i dati raw\footnote{I dati raw (o dati grezzi) sono l'insieme delle informazioni acquisite dai sensori non sottoposte ad alcun processo di filtraggio per la normalizzazione, mostrando il dato così come viene registrato.} ricevuti dai vari sensori, verificandone il corretto funzionamento e rendendo possibile un'analisi ancora più approfondita dei dati. Per poter però utilizzare Sensoria Lab, l'applicazione richiede l'acquisto dell'Software Development Kit\footnote{L'SDK è un pacchetto di strumenti per sviluppatori e ricercatori che utilizza le librerie ufficiali del produttore per integrare e raccogliere dati.} alla cifra di 999\$ l'anno per utente.

Un altro importante prodotto sul mercato delle applicazioni per sensori, è sicuramente \textit{Coospo Ride}. Questa app, incentrata sull'analisi della velocità dell'atleta e del suo battito cardiaco, permette di collegarsi ai dispositivo Coospo per poter poi procedere al monitoraggio. In questo caso i sensori sono meno complessi di quelli marchiati Sensoria, trattandosi prevalentemente di cardio-frequenzimetri talvolta combinati a sensori GPS per il tracciamento delle varie statistiche derivate dalla velocità.

\begin{figure}[htbp]
\centering
\begin{tabular}{cc}
\includegraphics[width=0.22\textwidth]{assets/coospohw.png} &
\includegraphics[width=0.22\textwidth]{assets/coospobici.png} \\
(a) Fascia da polso Coospo HW807 & (b) Ciclocomputer Coospo BC107
\end{tabular}
\caption{Vari sensori per atleti del marchio Coospo.}
\label{SensoriCoospo}
\end{figure}

Tramite l'app si potranno collegare i dispositivi desiderati e, come su Sensoria Run (vista nella sezione \ref{statoarte}), una volta avviata l'attività si potranno osservare a schermo diverse informazioni diverse in base al tipo di dispositivo che stiamo usando. Nel caso di utilizzo di un fascia da polso Coospo, per esempio, nell'app potremo vedere il nostro attuale battito cardiaco, seguito dalla velocità a cui stiamo correndo o camminando, in aggiunta ad ulteriori informazioni sull'attività che stiamo svolgendo. In questo modo si potrà denotare come durante l'attività dell'atleta ci saranno diverse "zone" che racchiuderanno un range di battiti. In base alla zona in cui ci troveremo in un dato momento del nostro allenamento, corrisponderà una percentuale di sforzo a cui si dovrà arrivare per ottimizzare la performance sportiva. In alcuni sensori come nell'HW807 (a sinistra della figura \ref{SensoriCoospo}) sarà visibile anche sul sensore stesso la zona tramite la colorazione di un led. Esistono anche altri tipi di sensori più complessi, come il ciclocomputer presente a destra della figura \ref{SensoriCoospo}, dal quale si potranno ottenere molte più informazioni osservandolo rispetto ad una fascia da polso. Una volta registrate e salvate correttamente le attività sarà possibile consultarle unicamente in app, potendo scorrere e analizzare diverse informazioni utili come si vede nell'immagine \ref{schermateCoospo}.

\begin{figure}[h]
    \centering
    \includegraphics[width=0.85\textwidth]{assets/schermataCoospo.png}
    \caption{Schermate di riepilogo attività dell'app Coospo Ride.}
    \label{schermateCoospo}
\end{figure}

Tra queste informazioni si può trovare:
\begin{itemize}
    \item La distanza percorsa.
    \item La velocità media su tutto il percorso.
    \item Il battito medio.
    \item Le calorie bruciate.
    \item Quanto tempo siamo restati in ogni "zona" di bpm.
    \item Numerosi grafici illustranti le varie misurazioni.
    \item Una mappa per visualizzare graficamente il percorso.
\end{itemize}

\section{Sviluppo ad-hoc}
Considerate le soluzioni analizzate nello stato dell'arte, è emersa la necessità di ideare un sistema ad-hoc per implementare delle funzionalità assenti nelle applicazioni ad oggi esistenti. Una di queste è la possibilità di poter osservare in modo più generico la mappa di un determinato percorso compiuto, colorando la mappa in base al battito registrato in un determinato settore, per dare un feedback immediato all'analizzatore delle diverse zone componenti il tracciato. Se si parla inoltre di proporre una soluzione strettamente professionale e non amatoriale, la scelta ottimale sarebbe esentare l'atleta da qualsiasi compito se non quello di avviare il tracciamento via app, scindendo dunque le funzionalità dell'app mobile, strettamente correlate all'avvio e all'arresto dell'attività, da quelle della dashboard web riservata all'addetto del monitoraggio, che comprenderanno l'analisi e la pulizia dei dati, creando statistiche utili e mostrando completamente ogni tipo di dato disponibile (compresi i dati raw). Un'ulteriore aspetto negativo nei prodotti presentati è la complessità dell'interfaccia in app, che porta l'atleta, non sempre esperto nell'ambito tecnologico, a confondersi e di conseguenza non riuscire nell'intento di collegare i sensori per avviare il monitoraggio dell'attività. Nella soluzione proposta e illustrata in questa tesi, l'interfaccia rimane minimale e intuitiva per permettere all'utente di capire istantaneamente come utilizzare l'applicativo, evitando di mantenere numerose e caotiche schermate che renderebbero caotico tutto il sistema. In entrambe le applicazioni proposte prima, la connessione ai sensori era strettamente riservata ai dispositivi dello stesso marchio dell'app, escludendo quindi di poter utilizzare, cardio-frequenzimetri di una marca e sensori di un'altra. Nella soluzione proposta invece, sono supportati tutti i tipi di dispositivi del marchio Sensoria per i dati più complessi, in aggiunta a qualsiasi tipo di dispositivo per il rilevamento del battito cardiaco, sfruttando la lettura delle caratteristiche Bluetooth di ogni unità collegata allo smartphone via app. Un'importante funzionalità non presente nella maggioranza dei prodotti presenti sul mercato, è la possibilità di trattare le attività registrate come replay, potendo scorrere tramite una comoda barra, lungo tutta la durata dell'attività, percependo meglio ogni singola variazione di dati nei sensori e permettendo così un'analisi molto più approfondita e precisa.

\subsection{Analisi dei requisiti}
I requisiti del sistema sono stati suddivisi in funzionali e non, al fine di garantire un'utilizzo piacevole mantenendo il corretto funzionamento dei componenti software. I requisiti funzionali comprenderanno:

\begin{itemize}
    \item Creazione di un profilo utente con annessi dati personali.
    \item Connessione a sensori IMU e calzini sensorizzati Sensoria.
    \item Connessione a cardio-frequenzimetri.
    \item Avvio e arresto del monitoraggio via app.
    \item Memorizzazione remota sicura.
    \item Visualizzazione dei dati raw via dashboard web.
    \item Visualizzazione di angoli di piegamento vari e pressioni plantari.
    \item Visualizzazione dei Bongiorno Indexes per ogni tibia.
    \item Visualizzazione della posizione GPS dell'atleta.
    \item Visualizzazione di statistiche legate allo spostamento.
    \item Replay delle attività.
\end{itemize}

Passando invece ai requisiti non funzionali, il sistema dovrà essere:

\begin{itemize}
    \item Affidabile.
    \item Estremamente reattivo.
    \item Piacevolmente usabile.
    \item Sicuro per i dati.
    \item Scalabile.
    \item Intuitivo.
\end{itemize}

Il complesso di tutti questi requisiti rappresenta un tassello fondamentale per la corretta riuscita del progetto, garantendone un'elevata efficienza e proponendo un'alternativa solida alle soluzione disponibili attualmente sul mercato, implementando nuove funzionalità innovative.

\subsection{Casi di studio}
Qui elencati si possono trovare alcuni casi di studio che coprono quasi tutte le possibilità di utilizzo del sistema:
\begin{enumerate}
    \item Atleta durante l'allenamento: l'atleta si prepara per il monitoraggio accendendo tutti i sensori e verificandone il corretto funzionamento tramite la colorazione di un led. Più precisamente i sensori che utilizza comprendono un sensore IMU posto sopra al ginocchio, un sensore IMU posto sotto al ginocchio, un cardio-frequenzimetro a fascia o a pettorina, e due calzini sensorizzati. Prosegue dunque, indossando uno ad uno tutti i sensori tramite le apposite fasce o sostegni. Successivamente, avvia l'applicazione, seleziona il suo profilo precedentemente creato e si connette a tutti i sensori accesi, verificando l'avvenuto abbinamento tramite il cambio di colore del led. Nel caso in cui un determinato sensore sia invertito (per esempio un sensore solitamente indossato a sinistra viene indossato a destra) gli basta modificare il nome del sensore, ovviando alla scomodità di doverselo togliere e rimettere. Una volta accesi, indossati e connessi i sensori, l'atleta procede a premere il tasto per l'avvio del tracking e aspetta cinque secondi a gambe dritte e parallele per la corretta calibrazione. Terminata la calibrazione procede con la vera e propria attività, potendo  comodamente mantenere il telefono a schermo spento. Una volta conclusa l'attività, l'atleta riaccende lo schermo del telefono per premere il tasto di fine tracciamento. In questo modo la registrazione dei dati avviene in modo corretto, terminando così il compito dell'atleta.
    \item Allenatore in analisi post-attività: una volta terminata l'attività, l'allenatore si connette alla dashboard web per consultare tramite il menù l'attività che gli interessa, facilmente distinguibile dalla data in cui essa è stata registrata. Una volta scelta e scaricata l'attività tramite l'apposito tasto, l'allenatore osserva a sinistra della dashboard gli spostamenti dell'atleta tramite una mappa interattiva, arricchita da informazioni quali velocità, distanza e battito cardiaco in quel preciso istante. Tramite una barra scorrevole posta in fondo alla mappa può spostarsi lungo tutta la durata dell'attività a suo piacimento per analizzare meglio determinate zone. A destra della mappa interattiva, può invece osservare, in un determinato istante le pressioni plantari dell'atleta, i Bongiorno Indexes, gli angoli di inclinazione delle tibie e l'angolo di flessione del ginocchio. Dalla mappa inoltre, l'allenatore può subito comprendere meglio le zone di sforzo maggiore dell'atleta semplicemente osservando il colore dei vari tratti del percorso. Tratte le conclusioni a seguito della revisione dei dati, l'allenatore indica all'atleta le zone più critiche e gli errori effettuati più comuni, per riuscire a migliorare le sue performance e prevenire eventuali infortuni.
    \item Ricercatore in analisi dati raw: il ricercatore si connette alla dashboard e carica un'attività, potendo così osservare in fondo al sito tutti i dati non filtrati dei sensori connessi durante l'attività. Essi comprendono tutti i dati dei sensori IMU (accelerometro, giroscopio e magnetometro) su tutti gli assi (x, y e z). Nel caso dei sensori montati sulle tibie, si vedono anche le pressioni laterali e frontali dei calzini. Osservando questi dati, il ricercatore può controllare se alcuni sensori restituiscono outliers \footnote{Con il termine outliers ci si riferisce a dei valori anomali rispetto a quelli precedentemente osservati.} oppure se certi valori tendono a rimanere fissi anche durante il movimento (segno di potenziale malfunzionamento del sensore). Una volta osservati tutti i valori, il ricercatore, se ha la possibilità di accedere al codice sorgente, può modificare determinati indici a suo piacimento o crearne di nuovi in base alle sue esigenze.
\end{enumerate}

\subsection{Modello di sviluppo}
Per lo sviluppo del sistema proposto è stato utilizzato un modello di sviluppo iterativo agile \footnote{Il modello di sviluppo iterativo agile prevede dei brevi cicli di sviluppo, chiamati sprint, abbinati ad un continuo confronto con l'utente.}, garantendo così alta flessibilità, elevata aderenza alle esigenze e uno sviluppo più rapido. L'adozione di questo modello di sviluppo si è rilevata particolarmente adatta al contesto del progetto siccome i requisiti erano in continua evoluzione in relazione alle richieste dell'utente. Il continuo confronto con gli utilizzatori finali ha assicurato che l'usabilità e l'efficienza fossero costantemente in crescita, grazie a costanti feedback e test. Questi ultimi hanno anche prevenuto errori progettuali che avrebbero causato ulteriori problematiche durante lo sviluppo del sistema. Ogni nuova funzionalità importante, una volta terminata, veniva prontamente testata per verificarne il corretto funzionamento ed eventualmente correggerne i bug presenti, assicurando così di ottenere un prodotto funzionante e facilmente perfezionabile grazie all'indipendenza di ogni componente dalle altre. Un altro importante fattore che ha contribuito alla riuscita di questo progetto è stata la possibilità di testare sul campo le funzionalità implementate di tanto in tanto, garantendo così un feedback più reale e veritiero.

\subsection{Architettura della soluzione}
L'architettura della soluzione proposta è di tipo modulare, progettata per acquisire, salvare, elaborare e visualizzare numerose quantità di dati provenienti da sensori wearable. Il sistema è composto da più livelli, ciascuno responsabile di determinate funzionalità al fine di garantire scalabilià, usabilità e flessibilità. 
Per garantire ordine, solidità ed efficacia, lo schema secondo il quale la soluzione è stata proposta segue la figura \ref{flowchart} \footnote{Lo schema composto dalla linea intera indica l'architettura pensata per il monitoraggio live. L'alternativa a linea tratteggiata mostra invece l'architettura del monitoraggio post attività.}.

\begin{figure}[h]
    \centering
    \includegraphics[width=1\textwidth]{assets/FLOWCHART.png}
    \caption{Diagramma di flusso dell'architettura proposta.}
    \label{flowchart}
\end{figure}

Come si può considerare, la prima fase del flusso è banalmente l'accensione e il montaggio dei sensori, comprendenti al massimo un cardio-frequenzimetro due sensori IMU per il ginocchio e due calzini sensorizzati. Una volta accesi essi possono essere successivamente connessi tramite l'applicazione apposita, avviando conseguentemente il monitoraggio. Nella fase successiva avviene dunque la calibrazione, che consiste nel mantenimento di una posizione a gambe tese e parallele per cinque secondi, al fine di ottenere gli angoli corretti di piegamenti delle tibie e flessioni del ginocchio. In seguito alla calibrazione avverrà l'acquisizione dei dati dell'attività praticata dall'atleta. Dall'istante in cui l'atleta avvierà l'attività, tutti i dati relativi ad essa saranno visibili online in tempo reale grazie alla dashboard. Una volta terminato il monitoraggio tramite app, il log relativo all'attività verrà salvato su un server remoto, tramite il quale si potrà poi visualizzare con la funzionalità di replay tutti i dati delle attività passate. Con questo intuitivo flusso di funzionamento, si giunge quindi ad una soluzione minimale che permette di analizzare i dati in modi diversi a seconda delle necessità, mantenendo sempre un rendimento elevato e puntando alla semplicità d'uso per l'utente.

\subsection{Caratteristiche tecniche}



%% Fine dei capitoli normali, inizio dei capitoli-appendice (opzionali)
\appendix

%\part{Appendici}

\chapter{Titolo della prima appendice}
Sed purus libero, vestibulum ut nibh vitae, mollis ultricies augue. Pellentesque velit libero, tempor sed pulvinar non, fermentum eu leo. Duis posuere eleifend nulla eget sagittis. Nam laoreet accumsan rutrum. Interdum et malesuada fames ac ante ipsum primis in faucibus. Curabitur eget libero quis leo porttitor vehicula eget nec odio. Proin euismod interdum ligula non ultricies. Maecenas sit amet accumsan sapien.

%% Parte conclusiva del documento; tipicamente per riassunto, bibliografia e/o indice analitico.
\backmatter

%% Riassunto (opzionale)
%\summary
%Maecenas tempor elit sed arcu commodo, dapibus sagittis leo egestas. Praesent at ultrices urna. Integer et nibh in augue mollis facilisis sit amet eget magna. Fusce at porttitor sapien. Phasellus imperdiet, felis et molestie vulputate, mauris sapien tincidunt justo, in lacinia velit nisi nec ipsum. Duis elementum pharetra lorem, ut pellentesque nulla congue et. Sed eu venenatis tellus, pharetra cursus felis. Sed et luctus nunc. Aenean commodo, neque a aliquam bibendum, mauris augue fringilla justo, et scelerisque odio mi sit amet diam. Nulla at placerat nibh, nec rutrum urna. Donec ut egestas magna. Aliquam erat volutpat. Phasellus vestibulum justo sed purus mattis, vitae lacinia magna viverra. Nulla rutrum diam dui, vel semper mi mattis ac. Vestibulum ante ipsum primis in faucibus orci luctus et ultrices posuere cubilia Curae; Donec id vestibulum lectus, eget tristique est.

%% Bibliografia (praticamente obbligatoria)
\bibliographystyle{plain_\languagename}%% Carica l'omonimo file .bst, dove \languagename è la lingua attiva.
%% Nel caso in cui si usi un file .bib (consigliato)
\bibliography{thud}
%% Nel caso di bibliografia manuale, usare l'environment thebibliography.

%% Per l'indice analitico, usare il pacchetto makeidx (o analogo).

\end{document}
